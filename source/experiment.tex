\documentclass[main.tex]{subfiles}

\begin{document}


The experimental apparatus involved two pendula (1 and 2) of masses $m_1 = 2931 g \pm 1 g$ and $m_2 = 2946 g \pm 1 g$  and lengths $L_1 = 84 cm \pm 1 cm$ and $L_2 = 83.5 cm \pm 1 cm$ respectively. Four different springs of different spring constant $k_i$ were used to couple the pendula (see Section IV for calculation of spring constants). We determined the spring constant for each spring by hanging different weights on each spring and measuring the subsequent extension. Plots of the hung mass vs the extension were used to calculate the spring constant.
\\
The pendula have 4 mounting points along their length for the spring (as described in the lab handout). The lengths of each of these mounting points were measured from the hinge of each pendulum.
\\
For analyzing normal modes and beats, we plot an amplitude vs time signal for each pendulum. This is done by means of a system consisting of a DC voltage generator, electrical contacts at the hinges of each pendulum and an analog to digital converter that is connected to a computer running a LabView tool that visualizes the signals.
\\
For normal modes, we plotted amplitude vs time signals for each pendulum for each initial condition (in-phase or out-of-phase) and every possible spring constant $k_i$ and mounting location $l_i$. We followed a similar arrangement for beat frequency, except that the initial conditions were changed such that pendulum 1 was held fixed at rest while pendulum 2 was displaced by some small angle. In both cases, we used a sampling rate of $5000 Hz$
\\
For odd normal modes, we made plots of the squared angular frequency $\omega^2$ vs $k_i$ and of $\omega^2$ vs $l_i$ (the coupling lengths) and used (6) to calculate the theoretical $\omega^2$ values.
\\
The theoretical beat frequency was calculated using (8).


\end{document}
