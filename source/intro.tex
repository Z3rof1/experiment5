\documentclass[main.tex]{subfiles}

\begin{document}

For small angles of displacement from its mean position, a simple pendulum behaves like a harmonic oscillator. When two such pendula are coupled (by means of a spring between them), it allows the transfer of energy between the two pendula and the force due to the spring changes the motion of the system. Each individual pendulum is now driven by the spring. As a result, the initial conditions i.e. the initial angular displacements of the pendula from their mean positions has an effect on their frequency $\omega_i (t)$ which depends now depends on the time. This is in contrast to the case of two uncoupled simple pendula, whose frequencies are time-independent. Two particular sets of initial conditions (see Section II) correspond to the 'normal modes' of the system. For these initial conditions, both pendula move sinusoidally with the same frequency and have a definite phase difference (Marion, J. & Thornton, S. 2004).


Coupled oscillators are used to model atoms in theories of radiation absorption/emission, which makes the study of coupled oscillators important. The normal modes of a coupled oscillator system are important because an arbitrary set of initial conditions of the system can be thought of as a superposition of the normal mode oscillations (Gardiner, J. 2013) (Marion, J. \& Thornton, S. 2004).


\end{document}